\documentclass{article}
\usepackage[authordate, backend=biber]{biblatex-chicago}
\usepackage[utf8]{inputenc}
\usepackage[T1]{fontenc}
\usepackage{lmodern}
\usepackage[ngerman]{babel}
\bibliography{literature} % or

\title{Die Bedeutung der lokalen Weißstorch-Population auf die Geburtenrate von Homo Sapiens Sapiens in Mitteleuropa}
\author{Sarah Köhler und das Chicago Manual of Style}
% \addbibresource{<database>.<extension>}
\begin{document}

\maketitle
\abstract
Dieser Artikel beleuchtet den Einfluss des Weißstorchs auf die Vermehrung
menschlicher Populationen.
Mit Hilfe verschiedener statistischer Verfahren (Korrelationsanalyse, Lineare Regression)
wird gezeigt, dass eine gesunde Storchenpopulation einen wichtigen Einflussfaktor
einer stabilen Geburtenrate in mitteleuropäischen Ländern darstellt.

\section{Vom Volksmund zur Forschungsfrage}
In zahlreichen mitteleuropäischen Ländern gibt es bis ins Mittelalter reichende Traditionen,
welche dem Weißstorch (Ciconia Ciconia) eine Rolle bei der Geburt menschlicher Säuglinge zuschreiben.
Einen erschöpfenden Überblick über die Spuren dieser Tradition in den volkstümlichen Medien
seit dem frühen Mittelalter bieten Mayer-Schönberger und Cukier \autocite[12]{mayer-schonberger_big_2013}.
Während diese populären Kulturartefakte jedoch zunehmend in Vergessenheit geraten,
mehren sich in der Wissenschaft die Stimmen, welche eine Untersuchung des realen Hintergrundes
dieser Überlieferungen fordern (vergleiche \cite[34]{orwat_software_2009} und \cite[167]{fleissner_heilung_2006}).
Diesbezüglich hat insbesondere die Aussage von Eckert ein großes Echo gefunden:
"`Die historisch-populäre Komparatistik beweist uns immer wieder,
dass es keiner noch so phantastischen Volkserzählung an einem wahren Kern mangelt."'\autocite[456]{eckert_mobil_2003}

\section{Ergebnisse der Datananalyse}

Um dem Zusammenhang zwischen der Populationsdichte von Ciconia Ciconia und der Geburtenrate von Homo Sapiens Sapiens mit statistischen Mitteln auf den Grund zu gehen, mussten zunächst die vorhandenen Daten in geeignete Regionen unterteilt werden.
Hierbei bildeten die Daten des IUCN zu den Beständen des Weißstorchs die Grundlage \autocite[76]{whitman_two_2004}.
Die Korrelationsanalyse zeigt deutlich, dass ein starker positiver Zusammenhang zwischen der Populationsdichte des Weißstorchs und der menschlichen Geburtenrate besteht.
Durch eine lineare Regression sowie eine Cluster-Analyse (n = 7) konnte zudem gezeigt werden, dass der Zusammenhang nicht linear verläuft: Die Populationsdichte von 50 Störchen auf 100 $km^2$ bildet eine obere Schranke des Zusammenhangs.
Auch der Vergleich mit historischen Daten bestätigt die Existenz einer Korrelation der Parameter \autocite[5-13]{coudert_when_2010}.

\section{Erklärungsansätze}

In der volkstümlichen Überlieferung transportiert der Storch neugeborene Kinder zum Haus ihrer Eltern \autocite[98f]{van_der_heijden_sedya:_2013}.
Durch Beobachtung von menschlichen Geburten konnte dieser Zusammenhang aber ausgeschlossen werden \autocite[3ff]{stopfer_personlichkeit_2010}.
Stattdessen muss die Nähe eines Storches auf andere Weise auf die Fortpflanzungsneigung wirken.
Denkbar wäre, dass die regelmäßige optische Wahrnehmung von Störchen im Landschaftsbild unterbewusste Prozesse anregt.
Einerseits könnte dies über die kulturell geprägte Identifikation des Storches mit Fruchtbarkeit und Fortpflanzung wirken.
Andererseits könnte der Storch auch als Repräsentant einer gesunden Kulturlandschaft fungieren und somit die Fortpflanzungsneigung indirekt beeinflussen,
indem die Lebensbedingungen und auch die Zukunftssicherheit positiver wahrgenommen werden.
Die unterbewussten Einflüsse könnten dann zum einen die Fortpflanzungneigung positiv verstärken.
Zudem ist auch eine psychosomatische Komponente anzunehmen, die etwa für eine höhere Erfolgsquote bei der Einnistung der befruchteten Eizellen sorgt.
Noch nicht ausgeschlossen werden konnte bisher auch die Theorie von Eli Pariser,
der davon ausgeht, dass der Grasfrosch (Rana temporaria) und speziell dessen Paarungsrufe die Fruchtbarkeit bei Menschen negativ beeinflussen \autocite[11f]{ted_eli_2011}.
Und da gesichert ist \autocite[666]{shapiro_information_2007}, dass Störche die Populationsdichte des Grasfroschs begrenzen, wäre somit auch ein indirekter Einfluss auf die menschliche Fortpflanzung anzunehmen.

\section{Ansätze für weitere Forschung}

Um festzustellen, in welcher Weise die Korrelation zwischen Populationsdichte des Weißstorchs und der menschlichen Geburtenrate zu erklären ist, ist weitere Forschungsarbeit notwendig.
In erster Linie sollte ein Vergleich mit anderen Regionen der Welt angestellt werden. Da Störche nicht weltweit verbreitet sind, wäre zu untersuchen ob eine ähnliche Korrelation auch mit anderen Tiergattungen zu beobachten ist.
Speziell zu betrachten wären Gattungen, die zum einen gut sichtbar, also tagaktiv und von einer gewissen Größe, sein sollten. Zum anderen müssen die Arten auch kulturelle Bedeutung aufweisen und dort positiv assoziiert sein.
Aber auch das Verhältnis von Weißstörchen und Menschen in Mitteleuropa sollte weiteren Untersuchungen unterzogen werden.

Genauere Vergleiche zwischen ländlichen und urbanen Populationen könnten Aufschluss geben, wie der Wirkzusammenhang funktioniert.
Zudem sollten auch die Grasfrosch-Vorkommen in die Betrachtung integriert werden.
Durch die genaue Beobachtung der Wanderungsbewegungen mittels Peilsendern könnte zudem erfasst werden, wie stark der Zusammenhang zeitlich gekoppelt ist.
Dazu wären auch Daten über die Winterwanderung der Störche in wärmere Gebiete einzubeziehen.


%Für Entscheidungsträger, welche für populationserhaltende Geburtenraten in Mitteleuropa sorgen möchten, ist es aber auch vor der Klärung des exakten Wirkungszusammenhangs angeraten, zu handeln.
%So sollten dringend Schutz- und Ansiedlungsprogramme aufgelegt werden, welche auf eine höhere Populationsdichte der Weißstörche abzielen.
%Notwendig wäre dazu vor allem der Schutz von Feuchtwiesen und Mooren. Aber auch eine Überprüfung der zugelassenen Insektizide und Herbizide um für eine größere Vielfalt an Futtertieren zu sorgen.



\printbibliography
\end{document}
