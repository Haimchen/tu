
\documentclass[11pt, oneside]{article}   	% use "amsart" instead of "article" for AMSLaTeX format
\usepackage{geometry}                		% See geometry.pdf to learn the layout options. There are lots.
\geometry{a4paper}                   		% ... or a4paper or a5paper or ... 
%\geometry{landscape}                		% Activate for for rotated page geometry
%\usepackage[parfill]{parskip}    		% Activate to begin paragraphs with an empty line rather than an indent
\usepackage{graphicx}				% Use pdf, png, jpg, or eps§ with pdflatex; use eps in DVI mode
								% TeX will automatically convert eps --> pdf in pdflatex		
\usepackage{amssymb}
\usepackage{amsmath}
\bigskip
\usepackage{booktabs}
% fï¿œr deutsche Zeichen
%\usepackage{ucs}
\usepackage[utf8]{inputenc}
\usepackage[T1]{fontenc}
%\usepackage[applemac]{inputenc}


\title{Arbeitsblatt 10}
\author{Dora Szücs und Sarah Köhler}
\date{}							% Activate to display a given date or no date

\begin{document}
\maketitle
%\section{}
%\subsection{}

\section*{Aufgabe 1.2 - Induktionsbeweis}
Zu zeigen: Das folgende Lemma ist korrekt:
Sei < $v_1, ...,  v_r$> der Inhalt der Schlange Q während eines Durchlaufs der Breitensuche auf einem Graph G = (V, E), wobei $v_1$ Kopf und $v_r$ Ende der Schlange ist. Dann gilt: \\
(a) $ d[v_r] \le d[v_1] + 1$ \\
und\\
(b) $ d[v_i] \le d[v_{i+1}]  $ für $ i = 1, 2, ..., r -1  $ \\

Wir beweisen per Induktion über die Schleifendurchläufe, d.h. nach n Schleifendurchläufen gilt die Behauptung.

\subsection*{I.A.}
Für $ n = 0$, d.h. vor dem ersten Schleifendurchlauf gilt:
Es befindet sich nur der Knoten s, der Startknoten, in der Schlange laut Zeile 1. Deswegen gilt \\
$ d[v_r] = 0 \le 1 = d[v_1] + 1$ \\
da sich nur ein Element in der Schlange befindet und somit $d[v_r] = d[v_1]$.
\\
Die zweite Bedingung ist für $n=0$ nicht erfüllbar, da nur ein Element in der Schlange ist und somit $d[v_i] = d[v_r] $ gelten müsste, was aber wegen des Definitionsbereichs von i nicht erlaubt ist ( $ i = 1, 2, ..., r -1  $).\\
 \\
Für $ n = 1$, d.h. nach dem ersten Schleifendurchlauf, muss man drei Fälle unterscheiden, je nach Anzahl der Nachbarn von Startknoten s. Hat s keine Nachbarn ist die Schlange nach dem ersten Durchlauf leer und der Algorithmus termininiert. Das Lemma trifft über eine leere Schlange keine Aussage, deswegen vernachlässigen wir diesen Fall. Hat s nur einen Nachbarn n, befindet sich nur dieser nach dem Schleifendurchlauf in der Queue und wir erhalten den selben Fall wie bei $ n = 0$.

Hat s mehrere Nachbarknoten, werden diese in der Schleife alle der Schlange hinzugefügt, d.h. nun befindet sich mehr als ein Knoten in der Schlange. Da s der Startknoten war und alle Knoten als weiß initialisiert wurden, müssen alle Nachbarn von s noch weiß sein. Es werden deswegen alle grau gefärbt (Zeile 5 und 6) und hinten an die Schlange angehängt (Zeile 9). \\
Für die Distanz eines beliebigen Nachbarn t von s gilt dann:\\
(*) $d[t] = d[s] + 1 = 0 + 1 = 1$\\
Der Startknoten s wird dagegen aus der Schlange entfernt (Zeile 12) und schwarz gefärbt.\\

In der Schlange befinden sich nun alle Nachbarn t des Startknotens s, somit gilt auch:\\
$ d[v_r] = d[v_1] \le d[v_1] + 1$ \\
und \\
$ d[v_i] \le d[v_{i+1}]  $ \\
da alle Knoten , die momentan in der Schlange sind, nach (*) die gleiche Distanz haben.\\

Für $ n = 1$ ist das Lemma also erfüllt.

\subsection*{I.V.}
Nach n Schleifendurchläufen gilt: \\
(1) $ d[v_r] \le d[v_1] + 1$ \\
(2) $ d[v_i] \le d[v_{i+1}]  $ für $ i = 1, 2, ..., r -1  $ \\

\subsection*{I.B.}

Nach $n+1$ Schleifendurchläufen soll gelten:  \\
(1) $ d[v_r] \le d[v_1] + 1$ \\
(2) $ d[v_i] \le d[v_{i+1}]  $ für $ i = 1, 2, ..., r -1  $ \\

\subsection*{I.S.}
Nach I.V. hat $v_1$ mindestens die Distanz $d[v_r] - 1$. \\
Da nach I.V. außerdem für alle Knoten gilt: $ d[v_i] \le d[v_{i+1}] $\\
haben die Knoten zwischen $v_1$ und $v_r$ die Distanz $d[v_r] - 1$ oder $d[v_r]$. \\

Dabei gilt: $v_r$ ist der letzte Knoten im vorhergehenden Schleifendurchlauf (der laut I.V. das Lemma erfüllt) und $v_r'$ der letzte Knoten im aktuellen Schleifendurchlauf ($n+1$).
\subsubsection*{Fall 1} $d[v_1] = d[v_r] - 1$\\
Die Knoten, die in die Schlange eingefügt werden, bekommen die Distanz\\
$d[v_1] + 1 = d[v_r] - 1 + 1 = d[v_r'] $.\\
Teil (a) des Lemmas gilt weiterhin, denn für den neuen Endknoten $v_r$ gilt: \\
$ d[v_r'] = d[v_1] + 1 \le d[v_1] + 1 $ \\
Teil (b) des Lemmas gilt auch, da die neuen Knoten $v_{r+1}$ bis $v_r'$ die selbe Distanz haben wie $v_r$.

\subsubsection*{Fall 2} $d[v_1] = d[v_r]$\\
Die neuen Knoten in der Schlange haben die Distanz \\
$d[v_1] + 1 = d[v_r] + 1 = d[v_r']$ \\

Teil (a) des Lemmas gilt, denn\\
$ d[v_r'] = d[v_1] + 1 \le d[v_1] + 1$\\
Auch Teil (b) des Lemmas gilt, da die Knoten von $v_1$ bis $v_r$ die gleiche Distanz haben ($d[v_r]$) und die Knoten von $v_{r+1}$ bis $v_r'$ alle die Distanz $d[v_r] + 1$ haben.\\
 \\

Für beide Fälle gilt, dass auch am Ende des Schleifendurchlaufs noch beide Teile des Lemmas erfüllt sind:
Wegen Zeile 12 wird $v_1$ entfernt.
Teil (a) gilt dann immer noch, denn $v_2$ ist nun das erste Element der Schlange und dafür gilt: \\
$d[v_2] = d[v_r']$ oder $d[v_2] = d[v_r'] - 1$\\
Beide Werte erfüllen (a).

Auch Teil (b) gilt dann immer noch, denn dies soll für $ i = 1, 2, ..., r -1$ gelten, also muss es ebenso für $ i = 2, 3, ..., r-1$ gelten.


\subsubsection*{Fazit} Somit lässt sich sagen, dass beide Teile des Lemmas in beiden Fällen gelten und aus I.A. und I.S. ergibt sich, dass das Lemma für alle Schleifendurchläufe erfüllt ist.



\section*{Aufgabe 3.3 - Kürzeste Wege mit Bellman-Ford}

\subsection*{Probleme}
Verändert man das Kantengewicht der Kante (D, B) zu 1, existiert ein Zyklus mit negativem Wert. Der Bellman-Ford-Algorithmus wird dann diesen Zyklus nicht mehr verlassen, da jeder Durchlauf die Gesamtkosten des Pfades weiter reduziert.

Allerdings kann man auch sagen, dass sich in einem Graphen, in welchem ein negativer Zyklus existiert, gar keine kürzesten Pfade bestimmen lassen. Somit würde Bellman-Ford nicht fehlschlagen sondern korrekterweise keine Lösung liefern, da keine existiert.

\subsection*{Aufwand im Vergleich zu Dijkstra}
Da in Dijkstras-Algorithmus jeder Knoten, der schwarz gefärbt wurde nicht mehr betrachtet wird, bewegt sich der Aufwand im Bereich $ \mathcal{O} (n)$ wobei n die Anzahl der Knoten im zu betrachtenden Graphen ist.
Dagegen überprüft der Bellman-Ford-Algorithmus auch die bereits besuchten Knoten und aktualisiert gegebenenfalls die Distanzen. Deswegen ist der Aufwand höher und liegt im Bereich von $ \mathcal{O} (n*v)$ wobei n wieder die Anzahl der Knoten ist und v die Anzahl der Kanten.

\end{document} 
