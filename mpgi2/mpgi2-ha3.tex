
\documentclass[11pt, oneside]{article}   	% use "amsart" instead of "article" for AMSLaTeX format
\usepackage{geometry}                		% See geometry.pdf to learn the layout options. There are lots.
\geometry{a4paper}                   		% ... or a4paper or a5paper or ... 
%\geometry{landscape}                		% Activate for for rotated page geometry
%\usepackage[parfill]{parskip}    		% Activate to begin paragraphs with an empty line rather than an indent
\usepackage{graphicx}				% Use pdf, png, jpg, or eps§ with pdflatex; use eps in DVI mode
								% TeX will automatically convert eps --> pdf in pdflatex		
\usepackage{amssymb}
\usepackage{amsmath}
\usepackage{booktabs}

% f�r deutsche Zeichen
%\usepackage{ucs}
\usepackage[utf8]{inputenc}
\usepackage[T1]{fontenc}
%\usepackage[applemac]{inputenc}


\title{Arbeitsblatt 3}
\author{Dora Sz�cs und Sarah K�hler}
\date{}							% Activate to display a given date or no date

\begin{document}
\maketitle
%\section{}
%\subsection{}

\section{Aufgabe 1.3 - Untere Schranken}
Sei 
\begin{equation*}
    f(n) = \frac{2}{n^3}(6n^4*\frac{\log 4n}{4}+n^3\log n)
\end{equation*}
 Zu zeigen:
\begin{equation}
  f(n)  \in  \Omega  (n log n)
\end{equation}
D.h. per Definition:
\begin{align*} 
f(n) = \frac{2}{n^3}(6n^4*\frac{\log 4n}{4}+n^3\log n)    \in   \Omega   (n \log n) \\
\Leftrightarrow \exists   c > n, n_0  \in   \mathbb{N}:  \forall  n \ge  n_0 : f(n) \ge c * (n \log n) \\
\end{align*}
Durch umformen erhalten wir:
\begin{align*} 
\frac{2}{n^3}(6n^4*\frac{\log 4n}{4}+n^3\log n)  \ge c * (n \log n)\\
\frac{2}{n^3} * \frac{1}{n \log n} (6n^4*\frac{\log 4n}{4}+n^3\log n) \ge c\\
\frac{2}{n^4 \log n} (\frac{6n^4 * \log 4n}{4}+n^3\log n) \ge c\\
\frac{2*6n^4 * \log 4n}{4n^4 \log n} + \frac{2n^3\log n }{n^4 \log n} \ge c\\
\frac{3 \log 4n}{\log n} + \frac{2}{n} \ge c\\
\frac{3 *(\log 4 + \log n}{\log n} + \frac{2}{n} \ge c\\
\frac{3 \log 4}{\log n} + \frac{3 \log n}{\log n} + \frac{2}{n} \ge c\\
\frac{3 \log 4}{\log n} + 3 + \frac{2}{n} \ge c\\
\end{align*}
Da f�r die Berechnung des Aufwandes die Entwicklung f�r gro�e Werte von n interessiert, berechnen wir den Grenzwert:
\begin{align*}
\lim_{n\rightarrow\infty}  ( \frac{3 \log 4}{\log n} + 3 + \frac{2}{n}) = 0 + 3 +0 = 3\\
\Rightarrow 3 \ge c
\end{align*}

Damit l�sst sich der Wert des Faktors c bestimmen und wir erhalten:
\begin{equation*}
\forall n \in \mathbb{N} .   n \ge n_0 : f(n) \ge c * (n \log n)  \mbox{ mit }  c < 3
\end{equation*}
Somit ist bewiesen, dass die Annahme korrekt war und es gilt $ f(n) \in \Omega (n \log n) $.


\section{Aufgabe 1.5 - Aufwandsberechnung}

F�r die gegebene Methode scheint der Best Case ein bereits sortiertes Array zu sein, der worst Case dagegen ein unsortiertes Array. In der Tabelle sind beide F�lle und der jeweilige Aufwand pro Anweisung angegeben:

% Requires the booktabs if the memoir class is not being used
\begin{table}[htbp]
   \centering
   %\topcaption{Table captions are better up top} % requires the topcapt package
   \begin{tabular}{@{} lcr @{}} % Column formatting, @{} suppresses leading/trailing space
      \toprule
      \multicolumn{2}{c}{Item} \\
      \cmidrule(r){1-2} % Partial rule. (r) trims the line a little bit on the right; (l) & (lr) also possible
      Animal    & Description & Price (\$)\\
      \midrule
      Gnat      & per gram & 13.65 \\
                & each     &  0.01 \\
      Gnu       & stuffed  & 92.50 \\
      Emu       & stuffed  & 33.33 \\
      Armadillo & frozen   &  8.99 \\
      \bottomrule
   \end{tabular}
   \caption{Remember, \emph{never} use vertical lines in tables.}
   \label{tab:booktabs}
\end{table}

\end{document}  
