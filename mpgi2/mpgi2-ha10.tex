
\documentclass[11pt, oneside]{article}   	% use "amsart" instead of "article" for AMSLaTeX format
\usepackage{geometry}                		% See geometry.pdf to learn the layout options. There are lots.
\geometry{a4paper}                   		% ... or a4paper or a5paper or ... 
%\geometry{landscape}                		% Activate for for rotated page geometry
%\usepackage[parfill]{parskip}    		% Activate to begin paragraphs with an empty line rather than an indent
\usepackage{graphicx}				% Use pdf, png, jpg, or eps§ with pdflatex; use eps in DVI mode
								% TeX will automatically convert eps --> pdf in pdflatex		
\usepackage{amssymb}
\usepackage{amsmath}
\bigskip
\usepackage{booktabs}
% f�r deutsche Zeichen
%\usepackage{ucs}
\usepackage[utf8]{inputenc}
\usepackage[T1]{fontenc}
%\usepackage[applemac]{inputenc}


\title{Arbeitsblatt 10}
\author{Dora Szücs und Sarah Köhler}
\date{}							% Activate to display a given date or no date

\begin{document}
\maketitle
%\section{}
%\subsection{}

\section*{Aufgabe 1.2 - Induktionsbeweis}
Zu Zeigen: Das folgende Lemma ist korrekt:
Sei < v1, ..., vr > der Inhalt der Schlange Q während eines Durchlaufs der Breitensuche auf einem Graph G = (V, E), wobei v1 Kopf und vr Ende der Schlange ist. Dann gilt d[vr] ≤ d[v1] + 1
und
d[vi] ≤ d[vi+1] für i = 1,2, ..., r − 1.

Wir beweisen per Induktion über die Schleifendurchläufe.

\subsection*{I.A.}

Sei 
\begin{equation*}
    f(n) = \frac{2}{n^3} * \left(6n^4*\frac{\log 4n}{4}+n^3\log n\right)
\end{equation*}
 Zu zeigen:
\begin{equation}
  f(n)  \in  \Omega  (n \log n)
\end{equation}
D.h. per Definition:
\begin{align*} 
& f(n) = \frac{2}{n^3} * \left(6n^4*\frac{\log 4n}{4}+n^3\log n\right()    \in   \Omega   (n \log n) \\
\Leftrightarrow \  &   \exists   c > n, n_0  \in   \mathbb{N}:  \forall  n \ge  n_0 : f(n) \ge c * (n \log n) \\
\end{align*}
Durch Umformen erhalten wir:
\begin{align*} 
\frac{2}{n^3}\left(6n^4*\frac{\log 4n}{4}+n^3\log n\right) & \ge c * (n \log n)\\
\frac{2}{n^3} * \frac{1}{n \log n} * \left(6n^4*\frac{\log 4n}{4}+n^3\log n\right) & \ge c\\
\frac{2}{n^4 \log n} * \left(\frac{6n^4 * \log 4n}{4}+n^3\log n\right) & \ge c\\
\frac{2*6n^4 * \log 4n}{4n^4 \log n} + \frac{2n^3\log n }{n^4 \log n} & \ge c\\
\frac{3 \log 4n}{\log n} + \frac{2}{n} & \ge c\\
\frac{3 *(\log 4 + \log n)}{\log n} + \frac{2}{n} & \ge c\\
\frac{3 \log 4}{\log n} + \frac{3 \log n}{\log n} + \frac{2}{n} & \ge c\\
\frac{3 \log 4}{\log n} + 3 + \frac{2}{n}  & \ge c\\
\end{align*}
Da für die Berechnung des Aufwandes die Entwicklung für große Werte von n interessiert, berechnen wir den Grenzwert:
\begin{align*}
\lim_{n\rightarrow\infty}  \left( \frac{3 \log 4}{\log n} + 3 + \frac{2}{n}\right) &  = 0 + 3 +0\\
& \Rightarrow 3 \ge c
\end{align*}

Damit lässt sich der Wert des Faktors c bestimmen und wir erhalten:
\begin{equation*}
\forall n \in \mathbb{N} .   n \ge n_0 : f(n) \ge c * (n \log n)  \mbox{ mit }  c < 3
\end{equation*}
Somit ist bewiesen, dass die Annahme korrekt war und es gilt $ f(n) \in \Omega (n \log n) $.


\section*{Aufgabe 3.3 - Kürzeste Wege mit Bellman-Ford}

\subsection*{Probleme}
Verändert man das Kantengewicht der Kante (D, B) zu 1, existiert ein Zyklus mit negativem Wert. Der Bellman-Ford-Algorithmus wird dann diesen Zyklus nicht mehr verlassen, da jeder Durchlauf die Gesamtkosten des Pfades weiter reduziert.

Allerdings kann man auch sagen, dass sich in einem Graphen, in welchem ein negativer Zyklus existiert, gar keine kürzesten Pfade bestimmen lassen. Somit würde Bellman-Ford nicht fehlschlagen sondern korrekterweise keine Lösung liefern, da keine existiert.

\subsection*{Aufwand im Vergleich zu Dijkstra}
Da in Dijkstras-Algorithmus jeder Knoten, der schwarz gefärbt wurde nicht mehr betrachtet wird, bewegt sich der Aufwand im Bereich $ \mathcal{O} (n)$ wobei n die Anzahl der Knoten im zu betrachtenden Graphen ist.
Dagegen überprüft der Bellman-Ford-Algorithmus auch die bereits besuchten Knoten und aktualisiert gegebenenfalls die Distanzen. Deswegen ist der Aufwand höher und liegt im Bereich von $ \mathcal{O} (n*v)$ wobei n wieder die Anzahl der Knoten ist und v die Anzahl der Kanten.

\end{document}  
