% !TEX encoding = UTF-8 Unicode

\documentclass[12pt]{amsart}
\usepackage{geometry} % see geometry.pdf on how to lay out the page. There's lots.
\geometry{a4paper} % or letter or a5paper or ... etc
% \geometry{landscape} % rotated page geometry
\usepackage{booktabs}


% See the ``Article customise'' template for come common customisations

\title{Hausaufgabe 4}
\author{Dora Szücs und Sarah Köhler}
\date{} % delete this line to display the current date

%%% BEGIN DOCUMENT
\begin{document}

\maketitle
%\tableofcontents

\section*{Aufgabe 4.1}
\subsection*{Algorithmus}

\begin{tabbing}
better(Points)\\
\>  distance $ \leftarrow  0$  \\
\> for i $ \leftarrow $ 1 To length(Points) do\\
\> \> distance $\leftarrow distance + \sqrt{X-Coord(Points[i])^2 + Y-Coord(Points[i])^2}\\
\> end for
\> distance $\leftarrow $ distance / length(Points) \\
\> if distance $ < $ 0.1 then \\
\> \> return true \\
\> else \\
\> \> return false \\
\> end if \\
Leave house. \\
\end{tabbing}

\subsection*{Korrektheitsbeweis}

\subsubsection*{Schleifeninvariante}

Zum Beweis der Korrektheit der Berechnungen innerhalb der for-Schleife dient die folgende Schleifeninvariante:


\fbox{%
  \parbox{0.9\textwidth}{%
  Zu Beginn jedes Schleifendurchlaufs enthält die Variable distance die Summe der  Abweichungen (vom Mittelpunkt) der ersten 1 bis (i -1) Punkte des Eingabearrays.}}

\section*{Aufgabe 4.2}

Zur Berechnung des Aufwandes betrachten wir die einzelnen, relevanten Zeilen des Algorithmus (in Pseudocode-Darstellung). Dabei ist n die Länge des Eingabearrays
% Requires the booktabs if the memoir class is not being used
\begin{table}[htbp]
   \centering
   %\topcaption{Table captions are better up top} % requires the topcapt package
   \begin{tabular}{@{} lcr @{}} % Column formatting, @{} suppresses leading/trailing space
      \toprule
     % \cmidrule(r){1-2} % Partial rule. (r) trims the line a little bit on the right; (l) & (lr) also possible
     Nr. & Code & Aufwand \\
      \midrule
      2      & distance $ \leftarrow $ 0  & 1 \\
      3       & for i $ \leftarrow $ 1 to length(Points) do     &  n \\
      4       & distance $ \leftarrow $ distance + $ \sqrt{X-Coord(Points[i])^2 + Y-Coord(Points[i])^2} $ & n \\
      6       & distance $ \leftarrow $ distance / length(Points)  & 1\\
      7 	& if distance $<$ 0.1 then   &  1 \\
      8 	& return true & 1 \\
      \midrule
        & Summe & 2n +4
      % \bottomrule
   \end{tabular}
   %\caption{Remember, \emph{never} use vertical lines in tables.}
   \label{tab:booktabs}
\end{table}
Aus der Tabelle ergibt sich ein gesamter Aufwand von $ 2n + 4 $, das bedeutet, dass der Aufwand der Funktion better in $ \Theta (n) $ liegt.
\end{document}