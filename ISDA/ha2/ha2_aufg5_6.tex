\documentclass[12pt]{amsart}

\usepackage[utf8]{inputenc}
\usepackage[T1]{fontenc}
\usepackage{lmodern}
\usepackage[ngerman]{babel}
\usepackage{graphicx}
%\usepackage{unicode-math}

\usepackage{geometry} % see geometry.pdf on how to lay out the page. There's lots.
\geometry{a4paper} % or letter or a5paper or ... etc
% \geometry{landscape} % rotated page geometry

% No indent at begin of paragraph
\setlength{\parindent}{0pt}
% line break after subsubsection heading
\makeatletter
\def\subsubsection{\@startsection{subsubsection}{3}%
  \z@{.5\linespacing\@plus.7\linespacing}{.1\linespacing}%
  {\normalfont\itshape}}
\makeatother
\usepackage{amssymb}
\usepackage{ifsym}
% define full outer join
\def\fullouterjoin{\tiny \textifsym{d|><|d}}

\title{Hausaufgabe 1 - Blatt 2, 3 \& 4}

%%% BEGIN DOCUMENT
\begin{document}
\subsection*{Aufgabe 5.1}

\subsubsection*{a) }
$ \pi_{Name, Vorname}(Angestellter \bowtie_{Pers.-Nr = Pers.-Nr}
(Betreut \bowtie_{Kunden-Nr = Kunden-Nr}(Firmenkunde))) $ \\

\begin{tabular}{|c|c|}\hline
Name & Vorname \\ \hline
Cartman & Eric \\
Bouvier & Selma \\
Simpson & Homer \\ \hline
\end{tabular}


\subsubsection*{b) }
$ A \cap B$ \\

mit \\
$ A = \pi_{Name}(Kunde \bowtie_{Kunden-Nr = Kunden-Nr}
(Bestellt \bowtie_{Produkt-Nr = Produkt-Nr} \linebreak
(\sigma_{Abteilung = ``NullBock``} (Produkt)))) $ \\

$ B = \pi_{Name}(Kunde \bowtie_{Kunden-Nr = Kunden-Nr}
(Bestellt \bowtie_{Produkt-Nr = Produkt-Nr} \linebreak
(\sigma_{Abteilung = ``NixDa``} (Produkt)))) $ \\

\begin{tabular}{|c|}\hline
Name   \\ \hline
Stefanie   \\ \hline
\end{tabular}


\subsubsection*{c) }

$ \pi_{Name, Vorname}(Angestellter \bowtie_{Pers.-Nr = Pers.-Nr}
(Betreut \bowtie_{Kunden-Nr = Kunden-Nr}(A)))$ \\
Mit\\
$ A = (Bestellt \bowtie_{Produkt-Nr = Produkt-Nr} Produkt) \backslash Abteilung $ \\

\begin{tabular}{|c|c|}\hline
Name & Vorname \\ \hline
Cartman & Eric \\ \hline
\end{tabular}

\subsection*{Aufgabe 5.2}

\subsubsection*{a) }

$ \pi_{Name, Bestellungen}(Kunde \fullouterjoin_{Kunden-Nr = Kunden-Nr} A) $ \\
Mit\\
$ A = \gamma_{Kunden-Nr, COUNT(Anzahl) \rightarrow Bestellungen}(Bestellt) $ \\

\begin{tabular}{|c|c|}\hline
Name & Bestellungen \\ \hline
Hella & 2 \\
Lukas & 1 \\
Stefanie & 3 \\
Thomas & - \\
Levin & 1 \\
Greta & 1 \\ \hline
\end{tabular}

\subsubsection*{b) }

$ \pi_{GehaltProStunde}
(\gamma_{(SUM(Gehalt) \medspace / \medspace SUM(hProWoche)) -> GehaltProStunde}(A)) $ \\
Mit\\
$ A = \sigma_{Abteilung=``NixDa`` \lor Abteilung=``Nicht \medspace in \medspace der \medspace Tabelle``}
(Angestellter)$

\begin{tabular}{|c|}\hline
2750  \\ \hline
\end{tabular}


\subsubsection*{c) }

$ \pi_{Name, Durchschnitt}(Kunde \bowtie_{Kunden-Nr = Kunden-Nr} A) $ \\
Mit\\
$ A = \gamma_{Kunden-Nr, AVG(Anzahl) \rightarrow Durchschnitt}
(\pi_{Kunden-Nr, Anzahl} (B))$ \\
$ B = Firmenkunde \bowtie_{Kunden-Nr = Kunden-Nr} Bestellt $ \\

\begin{tabular}{|c|c|}\hline
Name & Durchschnitt \\ \hline
Hella & 3.5 \\
Stefanie & 3.33 \\ \hline
\end{tabular}

\subsubsection*{d) }

$ \pi_{Bezeichnung, Produkt-Nr, Abteilung}
(Produkt \bowtie_{Produkt-Nr = Produkt-Nr} A) $ \\
Mit\\
$ A = \sigma_{Gesamtzahl \ge 3}(\gamma_{Produkt-Nr, SUM(Anzahl) \rightarrow Gesamtzahl}
(\pi_{Produkt-Nr, Anzahl}(Bestellt)))$

\begin{tabular}{|c|c|c|}\hline
Bezeichnung & Produkt-Nr & Abteilung \\ \hline
Aktenvernichter & 88 & NullBock \\
OUTDOOR Zelt & 100 & NixDA \\
1 TB HDD & 99 & Multimedia\\ \hline
\end{tabular}

Anmerkung: Wir haben die Aufgabenstellung so verstanden, dass von dem Produkt mindestens drei Stück verkauft werden sollen.

\subsection*{Aufgabe 5.3}

\subsubsection*{a) }
Der Name des Firmenkunden einer Firma vom Typ Mittel oder Klein, der die meisten Bestellvorgänge getätigt hat.

\begin{tabular}{|c|}\hline
Name \\ \hline
Stefanie  \\ \hline
\end{tabular}

\subsubsection*{b) }
Gesamtpreis aller Bestellungen, die der Kunde mit der Kunden-Nr 1 am 10.06.2013 durchgeführt hat.

\begin{tabular}{|c|}\hline
Betrag \\ \hline
400  \\ \hline
\end{tabular}

\subsection*{Aufgabe 6}

\subsubsection*{a) }
$ \tau_{P\_Name}(\pi_{P\_Name}(\sigma_{PS\_AvailQuantity > 0}
( A \bowtie_{P\_Partkey = PS\_Partkey} Partsupp))) $ \\
Mit\\
$ A = \pi_{P\_Partkey, P\_Name}
(\sigma_{P\_Brand=``Brand\#12`` \land P\_Size \le 19 \land P\_Size \ge 12 }(Part)) $ \\


\subsubsection*{b) }

$ \pi_{C\_Custkey, C\_Name, C\_Phone}(A) $ \\
Mit\\
$ A = \sigma_{N\_Name=``Germany`` \land C\_Accountbalance < 0}
(Customer \bowtie_{C\_Nationkey = N\_Nationkey} Nation)) $ \\
Anmerkung: Wir gehen davon aus, dass die Datenbank englischsprachige Datensätze enthält und im Attribut N\_Name die Bezeichnung für das Land gespeichert ist.\\


\subsubsection*{c) }

$ \tau_{Name, Phone}(A2 \cup B2) $\\
Mit\\
$ A = \pi_{C\_Name, C\_Phone}
(\sigma_{C\_Accountbalance \le 9000}(Customer)) $ \\

$ B = \pi_{S\_Name, S\_Phone}
(\sigma_{S\_Accountbalance \le 7000}(Supplier)) $ \\

$ A2 = \rho_{C\_Name \rightarrow Name, C\_Phone \rightarrow Phone}(A)) $ \\

$ B2 = \rho_{S\_Name \rightarrow Name, S\_Phone \rightarrow Phone}(B)) $ \\


\subsubsection*{d) }

$ \pi_{C\_Name, C\_Phone}
(Customer \bowtie_{C\_Custkey = Custkey} (B - A)))) $ \\
Mit\\
$ A = \rho_{O\_Custkey \rightarrow Custkey}(\pi_{O\_Custkey}(Orders)) $ \\

$ B = \rho_{C\_Custkey \rightarrow Custkey}(\pi_{C\_Custkey}(Customers)) $ \\
Anmerkung: Das Attribut C\_Phone entspricht dem in der Aufgabenstellung genannten C\_Phonenumber. Wenn das Attribut wirklich C\_Phonenumber heißen sollte, wäre noch eine Umbenennung erforderlich.

\subsubsection*{e) }

$ \pi_{C\_Name, C\_Phone}(Customer \bowtie_{C\_Custkey= O\_Custkey} C) $ \\
Mit\\
$ C = \pi_{O\_Custkey}(B \bowtie_{L\_Orderkey = O\_Orderkey} Orders) $ \\
$ B = A \backslash (\pi_{S\_Suppkey}(Supplier)) $ \\
$ A = \pi_{L\_Linenumber, L\_Orderkey, L\_Suppkey}(Lineitems \bowtie_{L\_Suppkey = S\_Suppkey} Suppliers) $ \\
Anmerkung: C\_Phone entspricht dem Attribut C\_Phonenumber aus der Aufgabenstellung.

\subsubsection*{f) }

$ \pi_{N\_Name}(B \bowtie_{S\_Nationkey= N\_Nationkey} Nation) $ \\
Mit\\
$ B = \sigma_{NumSuppliers \ge 17}
(\gamma_{S\_Nationkey, COUNT(S\_Supplykey) \rightarrow NumSuppliers)}(A))$ \\
$ A = \pi_{S\_Nationkey, S\_Supplykey}(Supplier) $ \\

\subsubsection*{g) }

$ \pi_{C\_Name}(D \bowtie_{O\_Custkey= C\_Custkey} Customer) $ \\
Mit\\
$ D = \sigma_{RevenuePerCustomer \ge 10000} (C) $ \\
$ C = \gamma_{O\_Custkey, SUM(L\_Extendedprice) \rightarrow RevenuePerCustomer}
(\pi_{O\_Custkey, L\_Extendedprice}(B)) $ \\
$ B = Orders \bowtie_{O\_Orderkey = L\_Orderkey}
(\pi_{L\_Linenumber, L\_ExtendedPrice, L\_Orderkey}(A)) $ \\
$ A = \sigma_{S\_Name = ``Supplier\#1``}(Lineitem \bowtie_{L\_Suppkey = S\_Suppkey}
(\pi_{S\_Suppkey, S\_Name}(Supplier))) $ \\


\subsubsection*{h) }

$ \pi_{C\_Name}(B \bowtie_{O\_Custkey= C\_Custkey} Customer) $ \\
Mit\\
$ B = \gamma_{O\_Custkey, MAX(Orders) \rightarrow MaxOrders}(A))$ \\
$ A = \gamma_{O\_Custkey, COUNT(O\_Orderkey) \rightarrow Orders}
(\pi_{O\_Custkey, O\_Orderkey}(Orders)) $ \\
\end{document}
