\documentclass[12pt]{amsart}

\usepackage[utf8]{inputenc}
\usepackage[T1]{fontenc}
\usepackage{lmodern}
\usepackage[ngerman]{babel}
\usepackage{graphicx}
% No indent at begin of paragraph
\setlength{\parindent}{0pt}
% line break after subsubsection heading
\makeatletter
\def\subsubsection{\@startsection{subsubsection}{3}%
  \z@{.5\linespacing\@plus.7\linespacing}{.1\linespacing}%
    {\normalfont\itshape}}
    \makeatother

\usepackage{geometry} % see geometry.pdf on how to lay out the page. There's lots.
\geometry{a4paper} %

\begin{document}
\section*{Aufgabe 5.2}

Belegungsmatrix $B$, Restanforderungsmatrix $R$ und freie Ressourcen $f$
für  Zeitpunkte $ t \in [1, 24]$ (Zustand jeweils zu Beginn der Zeitscheibe):

\subsubsection*{t = 1}

$ B =
\begin{pmatrix}
0 & 0 & 0 & 0 \\
0 & 0 & 0 & 0 \\
0 & 0 & 0 & 0 \\
0 & 0 & 0 & 0
\end{pmatrix},
R =
\begin{pmatrix}
2 & 4 & 3 & 0 \\
3 & 1 & 2 & 0 \\
3 & 0 & 4 & 0 \\
0 & 3 & 0 & 0
\end{pmatrix},
f =
\begin{pmatrix}
4 \\ % A
4 \\ % B
4 \\ % C
4    % D
\end{pmatrix} $ \\
sicherer Zustand mit
Ausführungsreihenfolge P1 - P2 - P3 - P4

\subsubsection*{t = 2}

$ B =
\begin{pmatrix}
% A & B & C & D
0 & 0 & 3 & 0 \\ % $P_1$
0 & 0 & 0 & 0 \\ % $P_2$
0 & 0 & 0 & 0 \\ % $P_3$
0 & 0 & 0 & 0    % P_4
\end{pmatrix},
R =
\begin{pmatrix}
2 & 4 & 0 & 0 \\
3 & 1 & 2 & 0 \\
3 & 0 & 4 & 0 \\
0 & 3 & 0 & 0
\end{pmatrix},
f =
\begin{pmatrix}
4 \\ % A
4 \\ % B
1 \\ % C
4    % D
\end{pmatrix} $ \\
sicherer Zustand mit
Ausführungsreihenfolge  P1 - P2 - P3 - P4


\subsubsection*{t = 3}

$ B =
\begin{pmatrix}
% A & B & C & D
0 & 4 & 3 & 0 \\ % $P_1$
0 & 0 & 0 & 0 \\ % $P_2$
0 & 0 & 0 & 0 \\ % $P_3$
0 & 0 & 0 & 0    % P_4
\end{pmatrix},
R =
\begin{pmatrix}
2 & 0 & 0 & 0 \\
3 & 1 & 2 & 0 \\
3 & 0 & 4 & 0 \\
0 & 3 & 0 & 0
\end{pmatrix},
f =
\begin{pmatrix}
4 \\ % A
0 \\ % B
1 \\ % C
4    % D
\end{pmatrix} $ \\
sicherer Zustand mit
Ausführungsreihenfolge  P1 - P2 - P3 - P4

\subsubsection*{t = 4}

$ B =
\begin{pmatrix}
% A & B & C & D
0 & 4 & 3 & 0 \\ % $P_1$
0 & 0 & 0 & 0 \\ % $P_2$
0 & 0 & 1 & 0 \\ % $P_3$
0 & 0 & 0 & 0    % P_4
\end{pmatrix},
R =
\begin{pmatrix}
2 & 0 & 0 & 0 \\
3 & 1 & 2 & 0 \\
3 & 0 & 3 & 0 \\
0 & 3 & 0 & 0
\end{pmatrix},
f =
\begin{pmatrix}
4 \\ % A
0 \\ % B
0 \\ % C
4    % D
\end{pmatrix} $ \\
sicherer Zustand mit
Ausführungsreihenfolge: P1 - P2 - P3 - P4


\subsubsection*{t = 5}

$ B =
\begin{pmatrix}
% A & B & C & D
0 & 4 & 3 & 0 \\ % $P_1$
0 & 0 & 0 & 0 \\ % $P_2$
0 & 0 & 1 & 0 \\ % $P_3$
0 & 0 & 0 & 0    % P_4
\end{pmatrix},
R =
\begin{pmatrix}
2 & 0 & 0 & 0 \\
3 & 1 & 2 & 0 \\
3 & 0 & 3 & 0 \\
0 & 3 & 0 & 0
\end{pmatrix},
f =
\begin{pmatrix}
4 \\ % A
0 \\ % B
0 \\ % C
4    % D
\end{pmatrix} $ \\
Allokation würde zu unsicherem Zustand führen, deswegen wird P3 blockiert und Zuweisung nicht ausgeführt.


\subsubsection*{t = 7}

$ B =
\begin{pmatrix}
% A & B & C & D
1 & 4 & 0 & 0 \\ % $P_1$
0 & 0 & 0 & 0 \\ % $P_2$
0 & 0 & 1 & 0 \\ % $P_3$
0 & 0 & 0 & 0    % P_4
\end{pmatrix},
R =
\begin{pmatrix}
1 & 0 & 0 & 0 \\
3 & 1 & 2 & 0 \\
3 & 0 & 3 & 0 \\
0 & 3 & 0 & 0
\end{pmatrix},
f =
\begin{pmatrix}
3 \\ % A
0 \\ % B
3 \\ % C
4    % D
\end{pmatrix} $ \\
P3 bleibt blockiert, sicherer Zustand mit
Ausführungsreihenfolge: P1 - P2 - P4 - P3



\subsubsection*{t = 10}

$ B =
\begin{pmatrix}
% A & B & C & D
1 & 4 & 0 & 0 \\ % $P_1$
0 & 0 & 2 & 0 \\ % $P_2$
0 & 0 & 1 & 0 \\ % $P_3$
0 & 0 & 0 & 0    % P_4
\end{pmatrix},
R =
\begin{pmatrix}
1 & 0 & 0 & 0 \\
3 & 1 & 0 & 0 \\
3 & 0 & 3 & 0 \\
0 & 3 & 0 & 0
\end{pmatrix},
f =
\begin{pmatrix}
3 \\ % A
0 \\ % B
1 \\ % C
4    % D
\end{pmatrix} $ \\
P3 bleibt blockiert, sicherer Zustand mit
Ausführungsreihenfolge: P1 - P2 - P4 - P3


\subsubsection*{t = 11}

$ B =
\begin{pmatrix}
% A & B & C & D
1 & 4 & 0 & 0 \\ % $P_1$
0 & 0 & 2 & 0 \\ % $P_2$
0 & 0 & 1 & 0 \\ % $P_3$
0 & 0 & 0 & 0    % P_4
\end{pmatrix},
R =
\begin{pmatrix}
1 & 0 & 0 & 0 \\
3 & 1 & 0 & 0 \\
3 & 0 & 3 & 0 \\
0 & 3 & 0 & 0
\end{pmatrix},
f =
\begin{pmatrix}
3 \\ % A
0 \\ % B
1 \\ % C
4    % D
\end{pmatrix} $ \\
Allokation kann nicht ausgeführt werden, da sonst ein unsicherer Zustand eintreten würde.
P2 blockiert, P3 bleibt blockiert.

\subsubsection*{t = 13}

$ B =
\begin{pmatrix}
% A & B & C & D
2 & 0 & 0 & 0 \\ % $P_1$
0 & 0 & 2 & 0 \\ % $P_2$
0 & 0 & 1 & 0 \\ % $P_3$
0 & 0 & 0 & 0    % P_4
\end{pmatrix},
R =
\begin{pmatrix}
0 & 0 & 0 & 0 \\
3 & 1 & 0 & 0 \\
3 & 0 & 3 & 0 \\
0 & 3 & 0 & 0
\end{pmatrix},
f =
\begin{pmatrix}
2 \\ % A
4 \\ % B
1 \\ % C
4    % D
\end{pmatrix} $ \\
P2 und P3 bleiben blockiert, sicherer Zustand mit
Ausführungsreihenfolge: P1 - P2 - P3 - P4


\subsubsection*{t = 14}

$ B =
\begin{pmatrix}
% A & B & C & D
2 & 0 & 0 & 0 \\ % $P_1$
0 & 0 & 2 & 0 \\ % $P_2$
0 & 0 & 1 & 0 \\ % $P_3$
0 & 3 & 0 & 0    % P_4
\end{pmatrix},
R =
\begin{pmatrix}
0 & 0 & 0 & 0 \\
3 & 1 & 0 & 0 \\
3 & 0 & 3 & 0 \\
0 & 0 & 0 & 0
\end{pmatrix},
f =
\begin{pmatrix}
2 \\ % A
1 \\ % B
1 \\ % C
4    % D
\end{pmatrix} $ \\
P2 und P3 bleiben blockiert, sicherer Zustand mit
Ausführungsreihenfolge: P1 - P2 - P3 - P4


Spätere Allokationsschritte werden nicht mehr ausgeführt, da P2 und P3 weiterhin blockiert sind.
\end{document}
