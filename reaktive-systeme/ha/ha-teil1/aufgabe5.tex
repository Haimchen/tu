\section*{Aufgabe 5: Starke Bisimulation}

\subsection*{d)}

Sei $\mathcal{B} = \{(q_2, p_1),(q_1, p_2),(q_4,p_2),(q_3,p_4),(q_5,p_3)\}$. \\
Beweis, dass $\mathcal{B} $ eine starke Bisimulation ist:

\begin{compactitem}
%% TODO, Fehler korrigieren! Erste zeile nach Aufgabenstellung (Grafik!) korrigieren!
%\item \ShowBisim{\MC{B}}{q_2,p_1}{{{a, q_3,p_3}{a,q_3,p_5},{a,q_2,p_5}}{{a, q_3,p_3}}
\item \ShowBisim{\MC{B}}{q_1,p_2}{{c,q_3,p_4},{b,q_5,p_3}}{{b, p_3,q_5},{c, p_4,q_3}}
\item \ShowBisim{\MC{B}}{q_4,p_2}{{b,q_5,p_3}}{{b, p_3,q_5}}
\item \ShowBisim{\MC{B}}{q_3,p_4}{{a,q_5,p_3}}{{a, p_3,q_5}}
\item \ShowBisim{\MC{B}}{q_5,p_3}{{a,q_2,p_1}}{{a, p_1, q_2}}
\end{compactitem}

Da $\mathcal{B}$ eine starke Bisimulation ist und $(q_2, p_1) \in \mathcal{B}$, gilt $q_2 \bisim p_1$.
\subsection*{a), b), c)}
Da $q_2 \bisim p_1$ (Beweis siehe d), gilt auch, dass $ q_2$ von $  p_1$ stark simuliert wird, sowie dass $p_1 $ von $ q_2$ stark simuliert wird. Damit simulieren sich $q_2$ und $p_2$ auch stark gegenseitig.
