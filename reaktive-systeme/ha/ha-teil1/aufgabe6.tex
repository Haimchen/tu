\section*{Aufgabe 6: Starke Bisimulation}

\subsection*{a)}
 $q_6$ und $p_1$ sind stark bisimilar. Eine passende Relation ist:

$\mathcal{B} = \{(p_1, q_6), (p_2, q_5), (p_3, q_9), (p_4, q_3), (p_5, q_3), (p_6, q_4), (p_6, q_7), (p_4, q_1), (p_5, q_2) \}$

\subsection*{b)} $q_6$  wird von $p_1$ stark simuliert, siehe Relation $ \mathcal{B} $ aus  a)

\subsection*{c)} $p_1$  wird von $ q_6$ stark simuliert, siehe Relation $ \mathcal{B} $ aus  a)

\subsection*{d)} $q_6$ und $ p_1$ simulieren sich gegenseitig stark, siehe Relation $ \mathcal{B} $ aus  a)

\subsection*{e)} $q_1 \bisim p_4$, d.h. es existiert eine starke Bisimulation:

Sei $\mathcal{R} = \{ (q_1, p_4), (q_3, p_5), (q_2, p_5), (q_4, p_6), (q_7, p_6), (q_8, p_4) \}$ \\
Beweis, dass $\mathcal{R}$ eine starke Bisimulation ist:

\ShowBisim{\MC{R}}{q_1,p_4}{{a,q_3,p_5},{b,q_2,p_5}}{{a, q_3,p_3}}

\ShowBisim{\MC{R}}{q_3,p_5}{{b,q_4,p_6},{b,q_7,p_6}}{{a, e, f}}

\ShowBisim{\MC{R}}{q_2,p_5}{{b,q_4,p_6}}{{b, p_6,q_4}}
Somit ist $\mathcal{R}$ eine starke Bisimulation. Da außerdem $(q_1, p_4) \in \mathcal{R}$ ist bewiesen, dass $q_1 \bisim p_4$ gilt.
