\subsection*{Aufgabe 3 - Charakteristische Formeln}

\subsubsection*{a)}

\begin{align*}
X_1 \HMmax& \necess{ \Act } \false \\
X_2 \HMmax& \poss{b} X_1 \land \necess{a}\false \land \necess{c}\false \land \necess{b}X_1 \\
X_3 \HMmax& \poss{a}X_5 \land \poss{b}X_7 \land \poss{c}X_6 \land \necess{a}X_5 \land \necess{b}X_7 \land \necess{c}X_6 \\
X_4 \HMmax& \poss{b}X_1 \land \poss{b}X_9 \land \necess{a}\false \land \necess{c}\false \land \necess{b}(X_1 \lor X_9) \\
X_5 \HMmax& \poss{a}X_2 \land \poss{a}X_4 \land \necess{b}\false \land \necess{c}\false \land \necess{a}(X_2 \lor X_4) \\
X_6 \HMmax& \poss{a}X_8 \land \necess{b}\false \land \necess{c}\false \land \necess{a}X_8 \\
X_7 \HMmax& \poss{a}X_8 \land \necess{b}\false \land \necess{c}\false \land \necess{a}X_8 = X_6\\
X_8 \HMmax& \poss{b}X_9 \land \necess{a}\false \land \necess{c}\false \land \necess{b}X_9 \\
X_9 \HMmax& \necess{ \Act } \false \\
\end{align*}

\subsubsection*{b)}

Es existieren die folgenden Äquivalenzklassen: \\

\begin{align*}
\necess{1}_\sim =& \{ 1, 9\} \\
\necess{3}_\sim =& \{ 3\} \\
\necess{5}_\sim =& \{ 5, 6, 7\} \\
\necess{2}_\sim =& \{ 2, 4, 8\} \\
\end{align*}
