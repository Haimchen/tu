\subsection*{Aufgabe 4 - min max mix}
Wir berechnen zuerst $\denot{Y}$: \\
Sei $F_Y = \poss{b} \true \lor ( \necess{ \Act} Y \land \poss{\Act} \true)$ \\

\begin{align*}
\OSem{F_Y}{\emptyset} =& \possDenot{b} \Proc \cup
(( \necessDenot{a} \emptyset \cap \necessDenot{b} \emptyset \cap \necessDenot{c} \emptyset) \cap
(\possDenot{a} \Proc \cup \possDenot{b} \Proc \cup \possDenot{c} \Proc)) \\
=& \{ p_3, p_6, p_9\} \cup \\
&(( \{p \in \Proc | p \NCCSTrans{a}\} \cap \{p \in \Proc | p \NCCSTrans{b}\} \cap \{p \in \Proc | p \NCCSTrans{c}\}) \cap \\
&( \{p \in \Proc | p \CCSTrans{a}\} \cup\{p \in \Proc | p \CCSTrans{b}\} \cup\{p \in \Proc | p \CCSTrans{c}\} )) \\
=& \{ p_3, p_6, p_9\} \cup (\{p \in \Proc | p \NCCSTrans{}\} \cap \{p \in \Proc | p \CCSTrans{}\} )\\
=&  \{ p_3, p_6, p_9\} \cup \emptyset \\
=&  \{ p_3, p_6, p_9\} \\
 (\OSem{F_Y}{})^2({\emptyset}) =& \possDenot{b} \Proc \cup \\
&((  \necessDenot{a}   \{ p_3, p_6, p_9\}  \cap  \necessDenot{b}   \{ p_3, p_6, p_9\}  \cap  \necessDenot{c}   \{ p_3, p_6, p_9\})  \cap \\
&(\possDenot{a} \Proc \cup \possDenot{b} \Proc \cup \possDenot{c} \Proc)) \\
=&  \{ p_3, p_6, p_9\} \cup \\
& (( \{ p_4, p_5, p_6, p_7, p_8, p_9\} \cap
\{ p_1, p_2, p_4, p_5, p_7, p_9\} \cap
\{ p_1, p_3, p_5, p_7, p_8\} \\
&( \{p \in \Proc | p \CCSTrans{a}\} \cup\{p \in \Proc | p \CCSTrans{b}\} \cup\{p \in \Proc | p \CCSTrans{c}\} )) \\
=&  \{ p_3, p_6, p_9\} \cup ( \{ p_5, p_7\} \cap \{p \in \Proc | p \CCSTrans{}\}) \\
=&  \{ p_3, p_6, p_9\} \cup \emptyset \\
=&  \{ p_3, p_6, p_9\}
\end{align*}

Da $\OSem{F_Y}{ \{ p_3, p_6, p_9\}} =  \{ p_3, p_6, p_9\}$ ist hier ein Fixpunkt (der kleinste) erreicht. \\
Wir berechnen nun $\denot{X}$ mit Hilfe des berechneten Fixpunktes. \\
Sei $F_X = \poss{a} Y \lor \necess{ \Act \setminus \{ a\}} X$. \\


\begin{align*}
\OSem{F_X}{\emptyset} =&
\possDenot{a} \{ p_3, p_6, p_8\} \cup
( \necessDenot{b} \emptyset \cap \necessDenot{c} \emptyset) \\
=& \{ p_1\} \cup
( \{ p_1, p_2, p_4, p_5, p_7, p_9\} \cap \{ p_1, p_3, p_5, p_7, p_8\} ) \\
=& \{ p_1\} \cup \{ p_1, p_5, p_7\} \\
=& \{ p_1, p_5, p_7\} \\
(\OSem{F_X}{})^2({\emptyset}) =&
\possDenot{a} \{ p_3, p_6, p_8\} \cup
( \necessDenot{b} \{ p_1, p_5, p_7\}  \cap \necessDenot{c} \{ p_1, p_5, p_7\} ) \\
=& \{ p_1\} \cup
( \{ p_1, p_2, p_4, p_5, p_7, p_8, p_9\} \cap \{ p_1, p_2, p_3, p_4, p_5, p_7, p_8, p_9\} )  \\
=& \{ p_1\} \cup
\{ p_1, p_2, p_4, p_5, p_7, p_8, p_9\} \\
=& \{ p_1, p_2, p_4, p_5, p_7, p_8, p_9\} \\
(\OSem{F_X}{})^3({\emptyset}) =&
\possDenot{a} \{ p_3, p_6, p_8\} \cup
(\necessDenot{b} \{ p_1, p_2, p_4, p_5, p_7, p_8, p_9\} \cap
\necessDenot{c} \{ p_1, p_2, p_4, p_5, p_7, p_8, p_9\} ) \\
=& \{ p_1\} \cup
( \{ p_1, p_2, p_3, p_4, p_5, p_6, p_7, p_8, p_9\} \cap
 \{ p_1, p_2, p_3, p_4, p_5, p_6, p_7, p_8, p_9\} )  \\
=& \{ p_1\} \cup \Proc \\
=& \Proc
\end{align*}

Da die Menge $\Proc$ nicht mehr größer werden kann und wir den maximalen Fixpunkt suchen, ist dieser hier erreicht. \\
Wir berechnen nun $\denot{Z}$ mit Hilfe des berechneten Fixpunktes. \\
Sei $F_Z = \poss{c} X \land \necess{\Act} Z $. \\

\begin{align*}
\OSem{F_Z}{\Proc} =&
\possDenot{c}\Proc \cap
( \necessDenot{a} \Proc \cap \necessDenot{b} \Proc \cap \necessDenot{c} \Proc) \\
=& \{p \in \Proc | p \CCSTrans{c}\} \cap
(\{p \in \Proc | p \NCCSTrans{a}\} \cap \{p \in \Proc | p \NCCSTrans{b}\} \cap \{p \in \Proc | p \NCCSTrans{c}\}) \\
=& \{p \in \Proc | p \CCSTrans{c}\} \cap
\{p \in \Proc | p \NCCSTrans{}\} \\
=& \emptyset
\end{align*}

Da die Menge nicht mehr kleiner werden kann und wir den minimalen Fixpunkt suchen, ist dieser erreicht. \\

Die Auswertung des gegebenen Gleichungssystems ergibt also die leere Menge, d.h. kein Prozess erfüllt die Formeln.

