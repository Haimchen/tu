\subsection*{Aufgabe 7 - Monotonie von O}


Zu zeigen: $\OSem{F}{}$ ist monoton $\forall F \in \henMil_{\{X\}}$. \\
Seien $S_1, S_2 \subseteq \Proc$ und es gelte $S_1 \subseteq S_2$.

Um die Monotonie von $\OSem{F}{}$ für beliebige $F$ zu beweisen, muss also gezeigt werden, dass
 $ \OSem{F}{S_1} \subseteq \OSem{F}{S_2}$ für alle $F \in \henMil$ gilt.

Beweis per struktureller Induktion über den Aufbau von $F$:

\subsubsection*{Induktionsanfang}
Induktionsanfang sind die atomaren Formeln $\true$ und $\false$: \\

\begin{align*}
\OSem{\true}{S_1} =& \Proc \subseteq \Proc  & Def. \OSem{}{}\\
=& \OSem{\true}{S_2} & Def. \OSem{}{}
\end{align*}

\begin{align*}
\OSem{\false}{S_1} =& \Proc \subseteq \Proc & Def. \OSem{}{} \\
=& \OSem{\false}{S_2} & Def. \OSem{}{}
\end{align*}

\subsubsection*{Induktionsvoraussetzung}

Seien $F, G \in \henMil$ so, dass $\OSem{F}{S_1} \subseteq \OSem{F}{S_2}$ und
$\OSem{G}{S_1} \subseteq \OSem{G}{S_2}$ gelten.



\subsubsection*{Induktionsbehauptung I}

Dann gilt auch:
$\OSem{F \lor G}{S_1} \subseteq \OSem{F \lor G}{S_2} $ \\

\subsubsection*{Induktionsschritt I}



\begin{align*}
\OSem{F \lor G}{S_1} =& \OSem{F}{S_1} \cup \OSem{G}{S_1}
 & Def. \OSem{}{}\\
=& \{ p \in \Proc | p \in \OSem{F}{S_1} \lor p \in \OSem{G}{S_1} \}
& Def. \cup \\
\subseteq &  \{ p \in \Proc | p \in \OSem{F}{S_2} \lor p \in \OSem{G}{S_2} \}
& (*) \\
=&  \{ p \in \Proc | p \in \OSem{F}{S_1} \} \cup
  \{ p \in \Proc | p \in \OSem{G}{S_1} \}
& Def. \cup \\
=& \OSem{F}{S_2} \cup \OSem{G}{S_2} & Def. \OSem{}{}\\
=& \OSem{F \lor G}{S_2} &
\end{align*}

$(*)$
Aus der I.V. folgt: \\
\begin{align*}
& p \in \OSem{G}{S_1} \Rightarrow  p \in \OSem{G}{S_2} \\
& p \in \OSem{F}{S_1} \Rightarrow  p \in \OSem{F}{S_2} \\
& \Rightarrow \{ p \in \Proc | p \in \OSem{F}{S_1} \lor p \in \OSem{G}{S_1} \}
\subseteq \{ p \in \Proc | p \in \OSem{F}{S_1} \lor p \in \OSem{G}{S_1} \}
\end{align*}


\subsubsection*{Induktionsbehauptung II}

Sei $a \in \Act$. Dann gilt auch:
$\OSem{\necess{a}F}{S_1} \subseteq \OSem{\necess{a}F}{S_2} $ \\

\subsubsection*{Induktionsschritt II}

\begin{align*}
\OSem{\necess{a}F}{S_1} =& \necessDenot{a}\OSem{F}{S_1}
& Def. \OSem{}{}\\
\OSem{\necess{a}F}{S_2} =& \necessDenot{a}\OSem{F}{S_2}
& Def. \OSem{}{}\\
\end{align*}

Es sind zwei Fälle zu unterscheiden: \\
I. Alle $p \in \Proc$ mit $ p \NCCSTrans{a}$. \\
Sei $P = \{p \in \Proc | p \NCCSTrans{s} \}$. \\
Nach Definition von $\necessDenot{a}$ sind alle Prozesse $p \in P$ in der Menge
$\OSem{\necess{a}F}{S_1} = \necessDenot{a}\OSem{F}{S_1}$ enthalten,
d.h. $P \subseteq \OSem{\necess{a}F}{S_1}$. \\
Analog gilt ebenso, dass alle $p \in P$ in der Menge
$\OSem{\necess{a}F}{S_2} = \necessDenot{a}\OSem{F}{S_2}$ enthalten sind, also
$P \subseteq \OSem{\necess{a}F}{S_2}$. \\

II. Alle $q \in \Proc$ mit $q \CCSTrans{a}$. \\
Für alle Prozesse q, die $a$-Übergänge haben, gilt nach Definition von $\necessDenot{a}$,
 dass sie dann in $\OSem{\necess{a}F}{S_1}$ enthalten sind,
wenn ein Übergang $q \CCSTrans{a} q'$ mit $q' \in \OSem{F}{S_1}$ existiert.
Sei $Q = \{ q \in \Proc | q \CCSTrans{a} q' \land q' \in \OSem{F}{S_1} \} $. \\
Es gilt also $Q \subseteq \OSem{F}{S_1}$. \\

Da nach I.V $\OSem{F}{S_1} \subseteq \OSem{F}{S_2}$ gilt, folgt daraus direkt, dass
$Q \subseteq \OSem{\necess{a}F}{S_2}$, da jeder $a$-Nachfolger, der in $\OSem{F}{S_2}$ enthalten ist, auch in $\OSem{F}{S_2}$ enthalten sein muss.


Da jeder Prozess aus $\OSem{\necess{a}F}{S_1}$ entweder in $P$ oder in $Q$
 enthalten sein muss, ist $\OSem{\necess{a}F}{S_1} = P \cup Q$.
Wir haben bereits gezeigt, dass $P \subseteq \OSem{\necess{a}F}{S_2}$ und
 $Q \subseteq \OSem{\necess{a}F}{S_2}$. Aus der Definition der Vereinigung folgt dann direkt,
 dass auch  $P \cup Q \subseteq \OSem{\necess{a}F}{S_2}$.
Es gilt also auch $\OSem{\necess{a}F}{S_1} \subseteq \OSem{\necess{a}F}{S_2}$.


\subsubsection*{Schluss}
Da wir gezeigt haben, dass die Behauptung für atomare Formeln $F$ gilt und induktiv auch über die Struktur von $F$, gilt die Behautung für alle Formeln $F \in \henMil$.
