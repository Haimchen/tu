\section*{Aufgabe 2: Vollständige Verbände}
Gegeben sind $ (X, R)$, wobei $X$ eine endliche Menge ist und $R$ eine Relation.

Zu zeigen: \\
$ (X,R) $ ist ein Verband $\Rightarrow (X, R) $ ist ein vollständiger Verband \\


Da $ (X, R) $ ein Verband ist, folgt aus der Definition eines Verbandes:\\
\begin{align*}
& \forall x,y \in X. \medspace\exists \Sup (\{x, y\}) \land \exists \Inf (\{x, y\})
\end{align*}

Um die Existenz des Infimums und Supremums für beliebige Teilmengen zu beweisen,
benötigen wir zunächst folgende Äquivalenz: \\
Seien $x, y, z \in X$ beliebig.
 Dann folgt aus der Definition eines Verbandes,
dass $(X, R)$ auch eine partiell geordnete Menge ist.
 Aus den Eigenschaften einer partiell geordneten Menge lässt sich ableiten, dass die Relation $R$ transitiv ist.
Deswegen muss gelten: \\
\begin{align*}
& \Sup (\{x, y, z\}) = \Sup( \{\Sup (\{x, y\}), z\}) &(*)\\
& \Inf (\{x, y, z\}) = \Inf( \{\Inf (\{x, y\}), z\}) &(**)
\end{align*}

Sei nun $ Y \subseteq X $ beliebig mit $Y = \{y_1, y_2, ... y_k\} $.\\
Aus der Endlichkeit von $X$ folgt, dass auch $Y$ endlich sein muss. \\

Für zwei beliebige Elemente $y_i$ und $y_j$ mit $ i,j \in [1,k] $ gilt wegen der Definition der Teilmengenbeziehung: \\
\begin{align*}
&y_i, y_j \in Y &\\
&\Rightarrow y_i, y_j \in X
& Definition \medspace von \medspace \subseteq\\
&\Rightarrow \exists z_i = \Sup (\{y_i, y_j\}) \land \exists z_s = \Inf (\{y_i, y_j\})
& X \medspace ist \medspace Verband\\
& \Rightarrow z_i, z_s \in X
& X \medspace Trägermenge \medspace von  \medspace (X, R) \\
\end{align*}


Für eine beliebige Teilmenge $Y \subseteq X$ lässt sich damit die Existenz des Supremums beweisen:

\begin{align*}
\Sup(Y) &= \Sup(\{y_1, y_2, ... , y_k\}) & \\
& =  \Sup(\{ \Sup(\{y_1, y_2\}), y_3, ... , y_k\}) & wegen  \medspace (*)\\
\end{align*}
Somit lässt sich das Supremum von $Y$ rekursiv bestimmen als:
\begin{align*}
\Sup(Y) &= \Sup(\{y_1, y_2, ... , y_k\}) & \\
& =  \Sup(\{ z_l, y_{l+2}, ... , y_k\}) & l < k+1, z_l = \Sup(\{z_{l-1}, y_{l+1}\}), z_l \in X\\
& =   \Sup(\{ z_{k-2}, y_k\}) & Y  \medspace endlich
\end{align*}

Da dies eine zweielementige Teilmenge von $X$ sein muss,
existiert auch ein Supremum. Somit ist auch die Existent eines Supremums der Menge $Y$ bewiesen.
Für die Vollständigkeit des Verbandes muss auch das Infimum von $Y$ existieren.
Analog zum Supremum gilt:

\begin{align*}
\Inf(Y) &= \Inf(\{y_1, y_2, ... , y_k\}) & \\
& =  \Inf(\{ \Inf(\{y_1, y_2\}), y_3, ... , y_k\}) & wegen  \medspace (**)\\
\end{align*}
Somit lässt sich das Supremum von $Y$ rekursiv bestimmen als:
\begin{align*}
\Inf(Y) &= \Inf(\{y_1, y_2, ... , y_k\}) & \\
& =  \Inf(\{ z_l, y_{l+2}, ... , y_k\}) & l < k+1, z_l = \Inf(\{z_{l-1}, y_{l+1}\}), z_l \in X\\
& =   \Inf(\{ z_{k-2}, y_k\}) & Y  \medspace endlich
\end{align*}

Dies ist wiederum eine zweielementige Teilmenge von $X$,
für die laut Definition ein Infimum existieren muss.
Damit ist auch die Existent des Infimums von $Y$ bewiesen.

Da also für eine beliebige Teilmenge von $X$ Infimum und Supremum existieren, ist $(X,R)$ ein vollständiger Verband.\\
$\Box$

