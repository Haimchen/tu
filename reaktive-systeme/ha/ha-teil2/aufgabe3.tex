\section*{Aufgabe 3: Bisimulation}

\begin{alignat*}{3}
\mathcal{F}^{1}(Proc \times Proc) &= r(s(\{&&(P_1, P_2),(P_1, P_5),(P_1, Q_1), (P_1, Q_2),(P_2, P_5),(P_2, Q_1),\\
								  &		  &&(P_2, Q_2),(P_3, Q_3),(P_4, Q_4),(P_5, Q_2),(P_5, Q_1),(Q_1, Q_2)\}))
\end{alignat*}
\\
$\mathcal{F}^{2}(Proc \times Proc) = r(s(\{(P_1, P_5),(P_1, Q_1),(P_2, Q_2),(P_3, Q_3),(P_5, Q_1)\}))$
\\
$\mathcal{F}^{3}(Proc \times Proc) = r(s(\{(P_2, Q_2),(P_5, Q_1)\}))$
\\
$\mathcal{F}^{4}(Proc \times Proc) = r(s(\{(P_5, Q_1)\}))$
\\
$\mathcal{F}^{5}(Proc \times Proc) = r(s(\{(P_5, Q_1)\}))$
\\\\
Da $\mathcal{F}^{4} = \mathcal{F}^{5}$ sind beide ein Fixpunkt.
\\
Somit erhalten wir, dass $P_5$ und $Q_1$ das einzige nicht trivial bisimilare Paar ist. Es gilt $P_5$ \bisim $Q_1$
