\section*{Aufgabe 4: Bisimulation (2)}

$\mathcal{F}^{1}(Proc \times Proc) = r(s(\{(R_1, R_7),(R_2, R_4),(R_2, R_6),(R_3, R_5),(R_4, R_6),(R_8, R_9)\}))$
\\
$\mathcal{F}^{2}(Proc \times Proc) = r(s(\{(R_2, R_6),(R_3, R_5)\}))$
\\
$\mathcal{F}^{3}(Proc \times Proc) = r(s(\{(R_2, R_6),(R_3, R_5)\})) = \mathcal{F}^{2}$
\\\\
Da $\mathcal{F}^{2} = \mathcal{F}^{3}$ sind beide ein Fixpunkt.
\\
Somit erhalten wir, dass $R_2$ und $R_6$ sowie $R_3$ und $R_5$ die einzigen nicht trivialen bisimilaren Paare sind. Es gilt $R_2$ \bisim $R_6$ und $R_3$ \bisim $R_5$.
