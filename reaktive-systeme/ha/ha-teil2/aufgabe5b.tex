\subsection*{b)}

Gegeben sind der vollständige, endliche Verband $(D, \sqsubseteq)$ sowie
die monotone Funktion $f: D \to D$.
Nach  Theorem 4.2 ist $z_{max}$, der größte Fixpunkt von $f$, wie folgt definiert:
\begin{align*}
&z_{max} = f^M(\top), M \in \mathbb{N}
\end{align*}

Zu zeigen: $z_{max}$ ist der größte Fixpunkt. \\
Dazu muss zunächst gezeigt werden, dass $z_{max}$ nach obiger Definition ein Fixpunkt ist
 und dann, dass es keinen größeren Fixpunkt gibt.

\subsubsection*{1. $z_{max}$ ist ein Fixpunkt}
Da $f$ laut Definition eine monotone Funktion ist und nichts größer sein kann,
als das größte Element, muss gelten:
\begin{align*}
& f(\top) \sqsubseteq \top
\end{align*}
Da $f$ total ist, kann die Funktion auch mehrfach angewendet werden.
Aus der Monotonie von $f$ und der Transitivität der Relation $\sqsubseteq$ folgt dann folgender Zusammenhang:
\begin{align*}
& f^k(\top) \sqsubseteq f^{k-1}(\top), k \in \mathbb{N}, k > 0
\end{align*}

Da in den Voraussetzungen $D$ als endlich gegeben ist, muss irgendwann ein Wert k erreicht werden,
wo der Zusammenhang konstant ist, wo also gilt:
\begin{align*}
& f^k(\top) = f^{k-1}(\top)
\end{align*}

Es muss also auch einen Wert $M$ geben, wobei für alle $k \leq M$ gilt:
\begin{align*}
& f^k(\top) = f^M(\top)
\end{align*}
Daraus lässt sich ableiten, dass ebenso gelten muss:
\begin{align*}
& f^M(\top) = f^{M-1}(\top) = f(f^{M-1}(\top))
\end{align*}
Daraus lässt sich direkt ablesen, dass $f^M(\top)$,
 also $z_{max}$ nach der Definition von Theorem 4.2,
 ein Fixpunkt von $f$ sein muss.

\subsubsection*{2. $z_{max}$ ist der größte Fixpunkt}
Sei $d \in D$ ein beliebiger Fixpunkt von $f$ mit $ d \ne f^M(\top)$.
Das bedeutet, dass $ d = f(d)$ gelten muss.
Es gilt außerdem, dass $d \sqsubseteq \top$, da $\top$ das maximale Element in $D$ ist.
Mit Hilfe der Monotonie von $f$ kann man folgendes folgern:
\begin{align*}
& d = f(d) \sqsubseteq f(\top)
\end{align*}
Das bleibt auch nach mehrmaliger Anwendung von $f$ gültig,
 so lange wie $f$ nicht mehr als $M$ mal hintereinander ausgeführt wurde.
Nach $M$ Schritten sind nach Definition die Fixpunkte erreicht und wir können keine
Aussage mehr über die Relation zu unserem beliebigen Fixpunkt $d$ treffen.

Als letzten gültigen Schritt erhält man also:
\begin{align*}
& d \sqsubseteq f^M(\top)
\end{align*}
Da somit alle anderen Fixpunkte $d$ kleiner sein müssen, können wir schließen,
dass $f^M(\top)$ der größte Fixpunkt von $f$ sein muss.
Somit ist bewiesen, dass $z_{max} = f^M(\top)$ der größte Fixpunkt von $f$ ist
 und diese Aussage von Theorem 4.2 gilt.\\
$\Box$
